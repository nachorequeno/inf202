\documentclass[11pt]{beamer}
\usetheme{CambridgeUS}
\usepackage[utf8]{inputenc}
\usepackage{amsmath}
\usepackage{blkarray}
\usepackage{amsfonts}
\usepackage{amssymb}
\usepackage{minted}

\usepackage{array,amsfonts}
\newcommand\myfunc[5]{%
  \begingroup
  \setlength\arraycolsep{0pt}
  #1\colon\begin{array}[t]{c >{{}}c<{{}} c}
             #2 & \to & #3 \\ #4 & \to & #5 
          \end{array}%
  \endgroup}


%\author{}
\title{Modélisation des structures informatiques: TP1}
%\setbeamercovered{transparent} 
\setbeamertemplate{navigation symbols}{} 
%\logo{} 
\institute{mouhcine.mendil@inria.fr} 
%\date{} 
%\subject{} 
\begin{document}

\begin{frame}
\titlepage
\end{frame}


%\begin{frame}
%\titlepage
%\end{frame}

%\begin{frame}
%\tableofcontents
%\end{frame}




\begin{frame}{Structure et Evaluation}
\begin{itemize}
\item 10 séances: 
\begin{itemize}
\item Exercices de TP
\item 2 projets
\end{itemize}
\onslide<2->{
\item Evaluation
\begin{itemize}
\item Soutenance de projets
\end{itemize}
}
\onslide<3->{
\item Objectif: Ecrire des programmes informatiques pour représenter les notions développées dans le cours
}
\end{itemize}
\end{frame}

\begin{frame}{Relations: Rappel}
\begin{definition}[Relation]
Une relation $R$ entre deux ensembles $A$ et $B$ est un sous-ensemble de $A \times B$.
\end{definition}
\onslide<2->{
\begin{example}[1]
$A=\{1,2\}$, $B=\{0,12,20\}$.  \newline
La relation $R$ entre $A$ et $B$ définie par $"\ge"$ est l'ensemble $\{(1,0),(2,0)\}$
\end{example} 
}
\end{frame}

\begin{frame}{Relations: Autre définition}
\begin{definition}[Relation]
Une relation $R$ entre deux ensembles $A$ et $B$ est une fonction $A \times B \rightarrow \{0,1\}$ telle que:
\begin{equation*}
\myfunc{f}{A \times B}{\{0,1\}}{(a,b)}{1 \textnormal{ si } aRb \textnormal{} \\ & & 0 \textnormal{ sinon}}
\end{equation*}
\end{definition}
\begin{columns}
\column{.45\textwidth}
\onslide<2->{
Représentation matricielle de $R$:
\begin{itemize}
\item $\forall i = 1,..,n \quad a_i \in A$
\item $\forall j = 1,..,m \quad b_i \in B$
\item $R[a_i][b_j] = 1$ ssi $a_iRb_j$
\end{itemize}
}
\column{.45\textwidth}
\onslide<3->{
\[
\begin{blockarray}{cccccc}
& b_1 & b_2 & .. & b_{m-1} & b_m \\
\begin{block}{c(ccccc)}
  a_1 & 1 & 0 & . & . & .\\
  a_2 & . & . & . & . & .\\
  ... & . & . & . & . & .\\
  a_n & . & . & . & . & .\\
\end{block}
\end{blockarray}
 \]
 }
\end{columns}
\end{frame}

\begin{frame}{Python: Rappels}
\begin{block}{}
\scriptsize
\inputminted{python}{src_python/tp1_cours.py}
\end{block}
\end{frame}

\begin{frame}{Structure du TP1}
\begin{itemize}
\item Partie 1: 
\begin{itemize}
\item Propriétés de l'anneau des booléens: élément neutre (1 ou vrai), élément absorbant (0 ou faux), la somme ($\vee$) et le produit ($\wedge)$
\item Affichage matricielle d'une relation 
\end{itemize}
\onslide<2->{
\item Partie 2: codage de relations simples (successeur, motié et diviseur non trivial)}
\onslide<3->{
\item Partie 3: Opérations sur les relations (symétrie, composition, union, intersection et inclusion)}
\onslide<4->{
\item Enoncé du TP et code à trous sur http://sergipujades.free.fr/teaching/INF202/}
\end{itemize}
\end{frame}

\end{document}